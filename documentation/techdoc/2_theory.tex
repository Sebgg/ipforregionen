\documentclass[techdoc/techdoc.tex]{subfiles}

\begin{document}

\section{Teori och referenser}
De resterande kapitlena av dokumentet är skrivna utifrån att läsaren är bekant
med openEHR och ramverket Angular. Om det inte är fallet följer i det här
kapitlet en kort beskrivning av viktiga koncept och referenser till mer
detaljerad dokumentation.

\subsection{openEHR}
OpenEHR är en väldigt stor standard men det här projektet behandlar endast en
liten del av den. Alla standardens olika delar går att läsa på
\url{https://specifications.openehr.org/}.
Koncept som är centrala för applikationen är framförallt kompositioner,
arketyper, \emph{templates} och elements datatyper.

Standarden är uppdelade i flera komponenter. Det här projektet behandlar
framförallt ``Reference Model'' (RM) och ``Archetype Model'' (AM). Mer
information om standardens arkitektur finns i openEHR:s
arkitekturöversikt.\footnote{
    \url{https://specifications.openehr.org/releases/BASE/latest/architecture_overview.html}
}

\paragraph{Arketyper} är datastrukturer som definierar hur all data ska
struktureras i en journal. Arketyper för noteringar (\emph{entry}) kan vara av
typen \emph{observable}, \emph{evaluation}, \emph{instruction} och
\emph{action}. Eftersom applikationen använder sig av självrapporterad data är
det endast \emph{observable} som används för noteringar. Ett exempel på en
\emph{observable} som används är ``Blood pressure''. Arketyper kan också
beskriva kompositioner. Arketyper definieras väldigt allmänt och är tänkta att
användas på global nivå. Arketyper definieras i AM.\footnote{
    \url{https://specifications.openehr.org/releases/AM/latest/Overview.html}
}

\paragraph{Kompositioner} är ett inlägg i en journal. En komposition består av
en kompositionarketyp som kan innehålla noteringar såsom ``Blood pressure''. I
applikationen används endast arketypen ``Self monitoring'' eftersom all data är
självrapporterad. Kompositioner defineras i RM.\footnote{
    \url{https://specifications.openehr.org/releases/RM/latest/ehr.html\#_compositions}
}

\paragraph{Templates} bestämmer vad en komposition kan och ska innehålla. Den
sätter ytterligare begränsningar på kompositionen än vad arketyperna gör.
Exempelvis har en \emph{template} använts i projektet som lägger till en
\emph{observable}-arketyp för varje tillgänglig kategori och därefter tar bort
alla element som inte är relevanta. \emph{Templates} definieras i AM.\footnote{
    \url{https://specifications.openehr.org/releases/AM/latest/AOM2.html\#_templates}
}

\paragraph{Datatyper} definierar hur individuella element av arketyper kan se
ut. Exempel på datatyper är \texttt{DV\_QUANTITY} och \texttt{DV\_TEXT} som
representerar kvantiteter och strängar som exempelvis en vikt eller en
kommentar. Datatyper är definerade i ``Data Types Information Model'' som är
del av RM.\footnote{
    \url{https://specifications.openehr.org/releases/RM/latest/data_types.html}
}


\subsection{Angular} \label{sec:ng}
Projektets applikation är byggd i ramverket Angular. Nedan följer en kort
beskrivning av de koncept från Angular som använts mycket i projektet.
Referenser till vidare dokumentation tillhandahålls också ifall mer information
eftersökes.

% TODO material biblioteket
% TODO moduler?

\subsubsection{Components och services}
Applikationer består av \emph{Components} eller komponenter som definierar
skärmelement som modifieras enligt komponentens logik. Varje komponent beskrivs
med hjälp av TypeScript, HTML och CSS. Applikationer har även \emph{services}
eller tjänster som komponenter kan använda för funktionalitet som inte är
direkt specifik till den komponenten, exempelvis HTTP-förfrågningar. Ett
exempel på en komponent i applikationen är \texttt{toolbar} som bestämmer hur
panelen i toppen av applikationen ska se ut och fungera. Ett exempel på en
tjänst är \texttt{EhrService} som hanterar kommunikation mot journalen.

Vidare information om komponenter och tjänster finns i Angulars officiella
dokumentation om dess arkitektur.\footnote{
    \url{https://angular.io/guide/architecture}
}

\subsubsection{Dependency Injection}
För att kunna använda en tjänst måste komponenten injicera den. Detta är ett
mönster inom Angular som kallas för \texttt{Dependency Injection}. Injicering
sker genom att lägga till tjänsten som ett argument till konstruktorn och
därefter tilldela den till en instansvariabel så att metoder kan komma åt den.
Mer information om mönstret finns i Angulars dokumention.\footnote{
    \url{https://angular.io/guide/dependency-injection}
}

\subsubsection{Filstruktur}
Angular har starka åsikter och konventioner. En av dessa konventioner som
upprätthålls av Angular och som utvecklare tvingas följa är applikationens
filstruktur.

Varje kompontent består av åtminstone tre filer; exempelvis
\texttt{toolbar.component.html},\\ % force newline to avoid overfull hbox
\texttt{toolbar.component.scss},
\texttt{toolbar.component.ts}
som innehåller komponentens HTML, CSS respektive TypeScript. CSS kan även
beskrivas med en \texttt{.css}-fil men i det här projektet används SASS.
Tjänster består av åtminstone en fil såsom
\texttt{ehr.service.ts}.

Nästan alla applikationens komponenter och tjänster har även en motsvarande
specifikationsfil såsom
\texttt{toolbar.component.spec.ts} och
\texttt{ehr.service.spec.ts}.
Dessa filer specificerar enhetstester för komponenten eller tjänsten.
Integrationstester definieras i mappen
\texttt{e2e/src}.

Mer detaljer om Angulars filstruktur finns i Angulars stilguide.\footnote{
    \url{https://angular.io/guide/styleguide}
}

\subsubsection{RxJS och Observable}
Angular använder sig av ett bibliotek som heter RxJS (Reactive Extensions for
JavaScript) för att hantera asynkron programmering. RxJS använder sig av så den
så kallade reaktiva programmeringsparadigmen. Projektet använder sig av
\texttt{Observable}-objekt från RxJS för att utföra HTTP-förfrågningar till
hälsoplattformar och till journalsystemet.

När en metod måste utföra en HTTP-förfrågan för att utföra sin uppgift
returneras alltid en \texttt{Observable}. Ett exempel är \texttt{getPartyId} i
\texttt{ehr/ehr.service.ts} som ska få tag på ett \texttt{partyId} utifrån ett
personnummer. Metoden utför inte sin förfrågan och returnerar strängen direkt.
Istället skapar den en \texttt{Observable} som anroparen kan prenumerera till
genom att anropa dess metod \texttt{subscribe}. Först när
\texttt{Observable}-objektet prenumereras på utförs förfrågan och strängen
returneras.

För mer information om RxJS hänvisas läsaren till Angulars introduktion till
RxJS\footnote{\url{https://angular.io/guide/rx-library}}
samt utvecklargruppens egna dokumentation som finns tillgänglig på deras
hemsida \url{http://reactivex.io/}


\end{document}
