\documentclass[techdoc/techdoc.tex]{subfiles}

\begin{document}

\section{Inledning} \label{sec:1_intro}
Detta dokument syftar till att ge en inblick i hur webbapplikationen för
inrapportering av hälsodata är strukturerad och vilka utvecklingsmöjligheter
som finns för produkten. Produkten är skapad av en grupp på sju studenter som
en del av deras kandidatarbete på uppdrag av Region Östergötland och har skett
i samråd med dem.

% TODO mer om bakgrund?

\subsection{Syfte}
Denna tekniska dokumentation ämnar att underlätta vid underhåll och
vidareutveckling av webbapplikationen för att rapportera in hälsodata. En
översiktlig arkitektur kommer presenteras som tolkningsstöd av koden och
specifika exempel för vidareutveckling kommer beskrivas. Dessa exempel baseras
på vad den ursprungliga utvecklingsgruppen ansåg vara mest centralt vid
vidareutveckling.
%och vara centrala för att följa applikationens ordinarie syfte.

Den tänkta målgruppen för detta dokument är utvecklare som inte tidigare
arbetat med applikationen. Därför läggs fokus först på att ge en överblick som
ska tillåta läsaren att snabbt sätta sig in i strukturen som beskrivs i sektion
\ref{sec:3_overview}. I sektion \ref{sec:5_develop} följer sedan konkreta
guider av de steg i vidareutvecklingsprocessen som originalutvecklarna anser
behöver en djupare beskrivning.

\subsection{Installation och användning}
Applikationen är byggd med ramverket Angular och använder därmed dess
uppsättning av verktyg känt som \emph{Angular CLI} för byggning och testning.
Applikationen använder \emph{npm} för att hantera paket.  Kommentarer i koden
är formaterade så att TypeDoc går att använda för att generera dokumentation.

\subsubsection{Bygga applikationen}
För att först installera alla paket som behövs körs kommandot
\begin{lstlisting}[language=sh]
    npm install
\end{lstlisting}
i rotmappen av projektet. Det skapar en mapp \texttt{node\_modules} som
innehåller alla dependencies till applikationen.

Därefter kan \emph{Angular CLI} användas för att kompilera applikationen. För
att skapa en distruerbar kompilerad version används kommandot
\begin{lstlisting}
    ng build
\end{lstlisting}
i rotmappen. Då skapas en kompilerad variant och läggs i mappen \texttt{dist}
med endast statiska filer.

\subsubsection{Utveckling}
För utveckling och felsökning går det att använda kommandot
\begin{lstlisting}
    ng serve
\end{lstlisting}
som bygger applikationen och serverar den på \texttt{localhost:4200} så att
applikationen går att komma åt med en webbläsare. Om någon av filerna ändras
byggs applikationen om automatiskt och sidan laddas om.

Kommandot
\begin{lstlisting}
    ng test
\end{lstlisting}
används för att köra alla enhets- och integrationstester.

För mer övergripande tester körs kommandot
\begin{lstlisting}
    ng e2e
\end{lstlisting}
som används för att köra alla end-to-end tester.

För att visa och eventuellt fixa kodstilsfel används kommandot
\begin{lstlisting}
    ng lint --fix
\end{lstlisting}
i applikationens rotmapp. Kodens stilguide och lintingens regler konfigureras i
filen \texttt{tslint.json}

\subsubsection{Dokumentation}
Kommentarerna i koden kan genereras till dokumentation med hjälp av TypeDoc.
Detta utförs genom att anropa kommandot
\begin{lstlisting}
    npm run docs
\end{lstlisting}
i applikationens rotmapp. Därefter läggs dokumentationen i mappen \texttt{docs}
som kan läsas genom att öppna \texttt{docs/index.html} med en webbläsare.
Konfiguration till TypeDoc finns i \texttt{typedoc.json} i projektets rotmapp.
% TODO hur gör man för att generera pdf?

\subsubsection{Google API key}
För nuvarande används en API-nyckel som skapats på ett konto som specifikt
användes för olika testningssyften under projektet. Id:t för API-nyckeln
återfinns i \texttt{google-fit-config.ts}. Eftersom denna nyckel används för
att definiera vilka URL:er som är auktoriserade att användas för hemsidan kan
det vara av intresse att skapa en ny API-nyckel där nya URL:er kan
auktoriseras. Detta görs enklast genom att följa stegen på
\url{https://developers.google.com/maps/documentation/javascript/get-api-key}.
Därefter måste \texttt{client\_id}-fältet i \texttt{google-fit-config.ts}
ändras till det genererade id:t.

\subsubsection{Konfiguration av EHR}
Filen \texttt{ehr/ehr-config.ts} innehåller ett EhrConfig-objekt som väljer
inställningar till journalsystemet. I objektet finns ett fält \texttt{baseURL}
som ska sättas till den URL som API-anropen för journalsystemet ska göras mot.
Det finns även ett fält \texttt{templateId} som specificerar vilken
\emph{template} som ska användas för att skapa kompositioner.

\end{document}
