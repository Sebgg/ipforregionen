\documentclass[techdoc/techdoc.tex]{subfiles}

% beskriva hur alla moduler fungerar, hur de är implementerade och varför

\begin{document}

\subsection{EHR: mot journal}
EHR-modulen hanterar framförallt inskickning av hälsodata till journalen. Den
innehåller också datastrukturer och funktioner som är baserade på
openEHR-standardens datatyper. Totalt innehåller modulen fyra filer:
\texttt{datatype.ts}, \texttt{datalist.ts}, \texttt{ehr.service.ts} och
\texttt{ehr-config.ts} som beskrivs i detalj i den här sektionen.


\subsubsection{datatype.ts} \label{sec:sys-datatype}
Filen \texttt{datatype.ts} definierar de datatyper som är baserade på openEHR.
Det finns två huvudsakliga objekt: den abstrakta klassen \texttt{DataType} och
gränssnittet \texttt{CategorySpec}.

\paragraph{DataType} definierar vad en datatyp ska innehålla och hur den
hanteras. Varje underklass motsvarar en DataType i openEHR.  Exempelvis
motsvaras \texttt{DV\_QUANTITY} och \texttt{DV\_DATE\_TIME} av underklasserna
\texttt{DataTypeQuantity} och \texttt{DataTypeDateTime}.  \texttt{DataType} har
metoder som \texttt{isValid} som avgör om ett visst värde är ett giltigt värde
för den datatypen. För \texttt{DataTypeQuantity} ser den till värdet är ett tal
inom det angivna intervallet. Datatypsklassen sköter även hur värden jämförs
med \texttt{equal} och \texttt{compare} samt hur värden konverteras till JSON
med \texttt{toRest}.

Instanser av datatypsklasser motsvarar ett element i en arketyp eller
kompositions-\emph{template}. Elementets egenskaper anges vid instansiering. Olika
datatyper kan ha olika egenskaper: \texttt{DataTypeCodedText} måste exempelvis
ha en lista av möjliga koder medan \texttt{DataTypeQuantity} måste ha en enhet
och ett intervall. Det finns också egenskaper som är gemensamma för alla
datatyper, bland annat namn, beskrivning och plats i \emph{template}. Alla gemensamma
egenskaper läggs först in i en \texttt{DataTypeSettings} och skickas därefter
till konstruktorn för datatypen tillsammans med de specifika egenskaperna.

Figur \ref{fig:dt_diastolic} visar ett exempel på en instansiering av elementet
för det diastoliska trycket inom blodtrycksarketypen. Här anges att elementet
ligger direkt under \texttt{any\_event}, borde visas som ``Diastoliskt'' samt
en beskrivning. Det anges också att elementet måste finnas, och att det kan
finnas fler än ett, samt att det ska vara synligt i tabellen för användaren.
Eftersom det är en kvantitet anges även att enheten är ``mm[Hg]'' och måste
ligga mellan 0 och 1000.

\begin{figure}[H]
    \lstinputlisting[language=java]{techdoc/lst/datatype_diastolic.ts}
    \caption{Instantiering av en datatype för diastoliskt tryck.}
    \label{fig:dt_diastolic}
\end{figure}

\paragraph{CategorySpec} är ett gränssnitt som motsvarar en \textit{template}
eller del av \textit{template} i openEHR. Varje instans av \texttt{CategorySpec}
representerar en kategori av hälsodata, såsom blodtryck. En instans innehåller
huvudsakligen en lista av instansierade datatyper som motsvarar element men även
namn, beskrivning och en identifierande sträng.

Ett exempel på en instansiering av en kategorispecifikation kan ses i figur
\ref{fig:spec_bp}. Här anges kategorins ID, namn, beskrivning samt ett antal
element som kategorin innehåller.

\begin{figure}[H]
    \lstinputlisting[language=java]{techdoc/lst/spec_bloodpressure.ts}
    \caption{Instantiering av en kategorispecifikation för blodtryck.}
    \label{fig:spec_bp}
\end{figure}

Notera att tid här tolkas som ett element, men det är egentligen del av
\texttt{Event}-objektet i arketypen. Alla kategorier innehåller därmed ett
element för tid. Detta är dels på grund av hur JSON-formatet för inskickning är
strukturerat, samt för att det är enklare att arbeta med generella element än
att hantera tid specifikt.


\subsubsection{datalist.ts}
Filen \texttt{datalist.ts} innehåller de två klasserna \texttt{DataPoint} och
\texttt{DataList}.

\paragraph{DataPoint} En instans av en datapunkt motsvarar ett mätvärde för
en kategori. Varje datapunkt innehåller ett värde för varje fält som kategorin
har. En datapunkt för blodtryckskategorin i figur \ref{fig:spec_bp} innehåller
därmed en tidpunkt, ett diastoliskt värde, ett systoliskt värde och en
position. Eftersom positionsfältet inte är markerat som \texttt{required} kan
det värdet vara tomt.

Implementationen av \texttt{DataPoint} är en endast en \texttt{Map} som mappar
fältens ID mot fältens värde. Klassen har även metoderna \texttt{equals} och
\texttt{compareTo} som jämför punktens alla fält med en annan punkts
motsvarande fält.

\paragraph{DataList} är en lista av datapunkter. En \texttt{DataList} kan
därmed tolkas som en stor matris eller tabell av mätvärden. Varje datalista har
en kategorispecifikation som definerar vilka fält som listan kan innehålla.
Klassen är enkapsulerad och klienter måste använda metoder för att komma åt
eller modifiera dess innehåll. Punkter läggs till med \texttt{addPoint} eller
\texttt{addPoints} och deras giltighet verifieras enligt kategorins datatyper.

En datalista har även egenskaper som \texttt{width} och \texttt{mathFunction}
som bestämmer hur värdena ska filtreras eller sammansättas. Om exempelvis
\texttt{width} är \texttt{DAY} och \texttt{mathFunction} är \texttt{MEAN} så
sammanställs alla punkter för varje dag till det genomsnittliga värdet för den
dagen.

Implementationen av \texttt{DataList} innehåller två listor vilket möjliggör
att filtreringar kan återställas. Instansens fält \texttt{points} innehåller de
ursprungliga punkterna som lagts till medan \texttt{processedPoints} innehåller
filtrerade punkter. Listan returnerar endast de behandlade punkterna med
metoden \texttt{getPoints} medan de ursprungliga punkterna ej är åtkomliga
utifrån.


\subsubsection{EHR service}
EHR-tjänsten agerar som en fasad mot EHR-modulen. Det finns två uppgifter för
EHR Service: tillhandahålla kategorispecifikationer för programmet samt skicka
in hälsodata till journalen.

\paragraph{Kategorispecifikationer} hämtas genom att först anropa metoden
\texttt{getCategories} för att se vilka kategorier som finns tillgängliga.
Därefter hämtas kategorispecifikationen för en given kategori med hjälp av
\texttt{getCategorySpec}.

För tillfället finns samtliga implementerade kategorier i
\texttt{ehr-config.ts} och hämtas helt enkelt därifrån.

\paragraph{Inskickning} utförs genom att först autentisera med
\texttt{authenticateBasic} och därefter anropa \texttt{sendData} med en
\texttt{DataList} och personnummer som inparameter och prenumerera till
\texttt{Observable}-objektet som den returnerar. \texttt{sendData} skapar en
JSON-formaterad komposition utifrån datalistan med hjälp av bland annat
datatypernas \texttt{toRest}-metoder.

\texttt{Observable}-objektet som \texttt{sendData} returnerar utför totalt tre
anrop mot journalen. Först måste ett \texttt{partyID} hittas genom att söka på
individer som har det givna personnumret med en \texttt{GET}-förfrågan till
\texttt{/demographics/party/query}. Därefter kan ett \texttt{ehrId} fås genom
att göra en \texttt{GET}-förfrågan till \texttt{/ehr}. När ett \texttt{ehrId}
väl har mottagits går det att skicka den skapade kompositionen med en
\texttt{POST}-förfrågan till \texttt{/composition}.


\subsubsection{ehr-config.ts}
Filen \texttt{ehr-config.ts} innehåller alla inställningar som är specifika
till det journalsystem som används. Just nu inkluderar detta ID:t till den \emph{template}
som används för alla inskickade kompositioner, URL:en till journalens REST-API
samt \emph{template}:s struktur beskriven med instanser av \texttt{CategorySpec}.

\end{document}
