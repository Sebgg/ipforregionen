\documentclass[techdoc/techdoc.tex]{subfiles}

\begin{document}

\subsection{Platform: mot hälsoplattformar} \label{sec:platform}
Plattformsmodulen sköter all kommunikation mot hälsoplattformar. Dess
huvudsakliga uppgift är att hämta data från hälsoplattformar och konvertera det
till instanser av \texttt{DataPoint}.

Eftersom olika plattformar har olika API:er och dataformat måste varje
hälsoplattform implementeras separat. Varje implementation för en
hälsoplattform är en service som är en underklass till \texttt{Platform}.

\subsubsection{Platform service} \label{sec:sys-platform}
\texttt{Platform} är den superklass som definierar det gemensamma gränssnittet
för alla plattformar. Utåt finns det fyra metoder tillgängliga;
\texttt{signIn}, \texttt{signOut}, \texttt{getAvailable} och \texttt{getData}.

Metoden \texttt{signIn} är tänkt att autentisera användaren, och bör anropas
när användaren väljer en specifik plattform. Väljer användaren Google Fit
kommer en pop-up där användaren loggar in på Google och blir förfrågad om att
dela med sig av sin hälsodata. Först när denna autentisering är klar kommer
nästa sida att laddas. Med anledning av detta behov för att vänta in metodens
exekvering så är \texttt{signIn} just nu märkt med \texttt{async}-prefixet,
vilket innebär att det är en asynkron funktion.
När autentiseringen väl har skett kan \texttt{getAvailable} anropas för att
avgöra för vilka kategorier som användaren har tillgänglig hälsodata som går
att hämta. Slutligen kan \texttt{getData} användas för att hämta all hälsodata
från en viss kategori inom ett visst tidsintervall.

\texttt{Platform} innehåller ett fält \texttt{implementedCategories} som är
ämnat att innehålla plattformsspecifika detaljer om varje kategori som är
implementerad, exempelvis motsvarande id hos plattformen. Fältet består av en
\texttt{Map} som mappar varje kategori-id till ett
\texttt{CategoryProperties}-objekt. Det objektet mappar i sin tur olika
datatyper för de olika kategorierna till de funktioner som används för att
hämta ut och konvertera datan från HTTP-förfrågan till det interna formatet.

\subsubsection{Google Fit service}
Klassen \texttt{GfitService} är en implementation för hälsoplattformen Google
Fit. Utöver de fält som nämns i \ref{sec:sys-platform} så har
\texttt{GfitService} två mappar av intresse: \texttt{commonDataTypes} som
mappar vanligt förekommande datatyper såsom datum och enhet till den funktion
som konverterar dem till det interna formatet, samt
\texttt{categoryDataTypeNames} som mappar Googles namn på datatyper till de
interna namnen.

I konstruktorn initieras sedan \texttt{implementedCategories}, och url:erna för
de olika kategoriernas dataströmmar kopplas till ett
\texttt{CategoryProperties}-objekt. Förutom vilken ström data ska hämtas ifrån
så definieras även funktioner för att konvertera data till det interna formatet
här.

För att hämta hälsodata från Google Fit så används i ett första steg metadata
för användaren. Denna metadata hämtas och analyseras i \texttt{getAvailable}.
Från metadatan extraheras alla tillgängliga kategorier och använda enheter.
Efter detta kan \texttt{getData} användas för att hämta data för en
implementerad kategori via en HTTP-förfrågan. I samband med hämtning av
kategoridata kallas även funktionen \texttt{convertData}, som använder sig av
\texttt{CategoryProperties}-objektet för att konvertera all hämtad data till
det interna formatet.

% TODO

\end{document}
